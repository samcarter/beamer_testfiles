% https://tex.stackexchange.com/a/430526
\documentclass{beamer}
\usepackage{tikz}

\usetikzlibrary{chains}
\usetikzlibrary{calc}
\usetikzlibrary{decorations.pathmorphing}

\definecolor{burgundy}{HTML}{882426}

\tikzset{onslide/.code args={<#1>#2}{%
          \only<#1>{\pgfkeysalso{#2}}
          }}
\tikzstyle{highlight}=[burgundy,ultra thick]

\begin{document}

    \begin{frame}[fragile]{Frame 1}
        \begin{figure}
            %\begin{tikzpicture}[remember picture,overlay,level/.style={sibling distance=60mm/#1}]
            \begin{tikzpicture}[level/.style={sibling distance=60mm/#1}]
                \edef\sizetape{0.7cm}
                \tikzstyle{tmtape}=[decorate, decoration={random steps,segment length=0.8pt,amplitude=0.2pt},draw,minimum size=\sizetape,thick]
                \tikzstyle{tmtap}=[draw,minimum size=\sizetape,black!50]
                \begin{scope}
                        \node [circle,decorate, decoration={random steps,segment length=0.8pt,amplitude=0.2pt}, draw,very thick,color=burgundy] (v0){\texttt{1}}
                        child {node [circle, decorate, decoration={random steps,segment length=0.8pt,amplitude=0.2pt}, draw,very thick,color=burgundy] (v1) {\texttt{2}} 
                            child {node [circle, decorate, decoration={random steps,segment length=0.8pt,amplitude=0.2pt}, draw,very thick,color=burgundy] (v3) {\texttt{3}}
                                child {node [circle, decorate, decoration={random steps,segment length=0.8pt,amplitude=0.2pt}, draw] (v9) {\scriptsize{\texttt{4}}}}
                                child {node [circle, decorate, decoration={random steps,segment length=0.8pt,amplitude=0.2pt}, draw,very thick,color=burgundy] (v8) {\texttt{5}}}
                            }
                            child {node [circle,decorate, decoration={random steps,segment length=0.8pt,amplitude=0.2pt}, draw] (v4) {\scriptsize{\texttt{6}}}}
                          }
                        child {node [circle,decorate, decoration={random steps,segment length=0.8pt,amplitude=0.2pt}, draw,very thick,color=burgundy] (v2) {\texttt{7}}
                        child {node [circle,decorate, decoration={random steps,segment length=0.8pt,amplitude=0.2pt}, draw] (v5) {\scriptsize{\texttt{8}}}
                                child {node [circle,decorate, decoration={random steps,segment length=0.8pt,amplitude=0.2pt}, draw] (v7) {\scriptsize{\texttt{9}}}}}
                        child {node [circle,decorate, decoration={random steps,segment length=0.8pt,amplitude=0.2pt}, draw,very thick,color=burgundy] (v6) {\texttt{10}}}
                        };
                    \path[draw,very thick,onslide={<2>{highlight}}] (v0) -- (v1);
                    \path[draw,very thick,onslide={<2>{highlight}}] (v1) -- (v3);
                    \path[draw,very thick,color=burgundy] (v3) -- (v8);
                    %\path[draw,very thick] (v2) -- (v5);
                    %\path[draw,very thick] (v5) -- (v7);
                    %\path[draw,very thick] (v3) -- (v9);
                    \path[draw,very thick,color=burgundy] (v2) -- (v6);
                    \path[draw,very thick,color=burgundy] (v2) -- (v0);
                    %\path[draw,very thick] (v1) -- (v4);
                    \node[xshift=0.2cm,yshift=0.2cm] (w0) at (v0.north east) {\textit{-1}};
                    \node[xshift=0.2cm,yshift=0.2cm] (w8) at (v8.north east) {\textit{3}};
                    \node[xshift=0.2cm,yshift=0.2cm] (w3) at (v3.east) {\textit{9}};
                    \node[xshift=-0.2cm,yshift=0.2cm] (w1) at (v1.west) {\textit{-17}};
                    \node[xshift=0.2cm,yshift=0.2cm] (w2) at (v2.north east) {\textit{13}};
                    \node[xshift=0.2cm,yshift=0.2cm] (w6) at (v6.north east) {\textit{-2}};
                \end{scope}
                %\begin{comment}
                \begin{scope}[shift={($(v1.south)+(1,-4.5)$)},start chain=1 going right, node distance = 3.55mm]
                        \node [on chain=1,tmtape] (am17) {\texttt{-17}};
                        \node [on chain=1,tmtape] (am2) {\texttt{-2}};
                        \node [on chain=1,tmtape] (am1) {\texttt{-1}};
                        \node [on chain=1,tmtape] (a3) {\texttt{3}};
                        \node [on chain=1,tmtape] (a9) {\texttt{9}};
                        \node [on chain=1,tmtape] (a13) {\texttt{13}};
                        \coordinate (X) at ($(a13.south east)+(1,-1)$);
                        %\draw [decorate, decoration={brace,amplitude=10pt,mirror,raise=4pt},very thick,yshift=-2cm] (a8l.south  west) -- (a6f.south east) 
                    %   node[midway,yshift=-0.85cm] (hp0) {\footnotesize{\texttt{8-to-6 path}}};
                    %\draw [decorate, decoration={brace,amplitude=10pt,raise=4pt},very thick,yshift=-2cm] (a5f.north west) -- (a5l.north east) 
                    %   node[midway,yshift=0.85cm] (hp1) {\footnotesize{\texttt{omitted vertices}}};
                        \path [draw,dotted,thick,->,out=0,in=-45] (w8) -- ($(a3.north west)+(-0.07,0.07)$);
                        \draw [dotted,thick,->,bend right=90,in=-90,out=0] (w6.south) ..controls (X) .. ($(am2.south east)+(0.07,-0.07)$);
                \end{scope}
                %\end{comment}
            \end{tikzpicture}
            %\caption{Path median query example.}
            %\label{fig02}
        \end{figure}
    \end{frame}
\end{document}
