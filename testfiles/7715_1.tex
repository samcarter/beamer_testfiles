% https://texnique.fr/osqa/questions/7715/simplifier-les-notes-beamer-pour-impression/7717
\documentclass[french,12pt,]{beamer}
\setbeameroption{show notes}
\usepackage{pgfpages} 
\pgfpagesuselayout{2 on 1}[a4paper,border shrink=5mm]
\usepackage[utf8]{inputenc}
\usepackage[T1]{fontenc}
\usecolortheme[RGB={0,120,0}]{structure}
\useoutertheme{default}
\usepackage{tcolorbox}
\tcbuselibrary{skins}
\usepackage[french]{babel}
\title{Le titre de la présentation}
\subtitle{Avec un sous-titre}
\author{Joseph Tux}
\date{ECM du \today}

\setbeamertemplate{note page}{\insertnote}

\begin{document}
%\insertslideintonotes{0}
\setbeamertemplate{frametitle}[default][center]
\setbeamertemplate{navigation symbols}{}    % Pas de symboles de navigation

\begin{frame}[plain]
  \titlepage{}
\note{\large 
\vspace{-1\baselineskip}
\textcolor{blue}{Commandes de dspdfviewer}
\vspace{-0.5\baselineskip}
\begin{itemize}
    \setlength{\itemsep}{-1ex} 
  \item «F1»,«?»    Affiche l'aide,
  \item «H»         (Home) (tout repart à 0),
  \item «G \emph{p}» (Go) Va à la page \no{} \emph{p};
  \item «S», «F12»  (Switch) Intervertir les écrans public/conférencier,
  \item «T»         (Toggle) Affiche soit la diapos, soit le commentaire,
  \item «D»         (Duo) Affiche      la diapos  et   le commentaire,
  \item «B»         (Blanck) Éteint l'écran public 
\end{itemize}
}
\end{frame}

\begin{frame}{\fbox{Plan}}
\note{\large Chaque diapo peut-elle être l'occasion de dialoguer\\n'hésitez
pas à demander le micro pour intervenir après chaque diapositive}
  \begin{center}
    \begin{tcolorbox}[size=normal,fontupper=\small,width=0.68\textwidth,colframe=gray!40,colback=yellow!10,skin=beamer,beamer]
      \tableofcontents
    \end{tcolorbox}
  \end{center}
\end{frame}

\section{Section 1}
\subsection{Subsection 1}
\begin{frame}{Préliminaire} %N° 3
  \begin{block}{De quoi parle-t-on?}
    \begin{center}
      Du blabla pour l'exemple.
      \begin{itemize}
        \item Du blabla pour l'exemple.
          \note{Du texte dans la note.}
        \item Du blabla pour l'exemple.
          \note{Du texte dans la note.}
      \end{itemize}
        \end{center}
  \end{block}
\end{frame}

\subsection{Subsection 2}
\begin{frame}{Suite}
  \begin{block}{Subsection 2}
    \begin{center}
      Du blabla pour l'exemple.
      \begin{itemize}
        \item Du blabla pour l'exemple.
          \note{Du texte dans la note.}
        \item Du blabla pour l'exemple.
          \note{Du texte dans la note.}
      \end{itemize}
        \end{center}
  \end{block}
\end{frame}

\section{Section 2}
\subsection{Subsection 1}
\begin{frame}{Suite}
  \begin{block}{Subsection 2}
    \begin{center}
      Du blabla pour l'exemple.
      \begin{itemize}
        \item Du blabla pour l'exemple.
          \note{Du texte dans la note.}
        \item Du blabla pour l'exemple.
          \note{Du texte dans la note.}
      \end{itemize}
        \end{center}
  \end{block}
\end{frame}

\end{document}