% https://tex.stackexchange.com/a/365750
\documentclass[10pt]{beamer}

\usetheme{Warsaw}
\usepackage{beamerthemesplit}
\setbeamercovered{transparent}
\usepackage[english]{babel}


%\beamerdefaultoverlayspecification{<+->}

\setbeamertemplate{frametitle continuation}{}

\begin{document}

\begin{frame}[allowframebreaks]
  \frametitle{References:}
  \begin{enumerate}
        \bibitem{1Aki}
        J. Akiyama, T. Hamada, I. Yoshimura, On characterizations of the middle graphs, TRU Mathematics 11 (1975) pp, 35-39.

        \bibitem{2Alsp}
        B. Alspach, C.C.Chen, Kevin McAvaney, On a class of Hamiltonian laceable 3-regular graphs, Disc. Math. 151 (1996) pp 19-38.

        \bibitem{3Alsp}
        B. Alspach, C.Q. Zhang, Hamilton cycles in cubic Cayley graphs on dihedral groups, Ars Combin. 28 (1989), pp 101-108.

        \bibitem{4Anad}
        B.S. Anand, M. Changat, S. Klavzar, I. Peterin, Convex sets in lexicographic products of graphs, Graphs Combin. 28 (2012), 77-84.

        \bibitem{5Basa}
        M. Basavaraju, L.S. Chandran, D. Rajendraprasad, A. Ramaswamy, Rainbow connection number of graph power and graph products,  arXiv:1104.4190v1 [math.co] (2011).

        \bibitem{6Beh}
        M. Behzad, G Chartrand, Total graphs and traversability, Proc. Edinburgh Math. Soc. (2) 15 (1966/67), pp 117-120.

        \bibitem{7Beh}
        M. Behzad, A criterion for the planarity of the total graph of a graph, Proc. Cambridge Philos. Soc. 63 (1967), pp 679-681.

        \bibitem{8Beh}
        M. Behzad, The connectivity of total graphs, Austr. Math. Bull. 1 (1969), pp 175-181.

        \bibitem{9Beh}
        M. Behzad, a characterization of total graphs, Amer. Math. Soc. 26 (3), (1970), pp 383-389.

        \bibitem{10Benk}
        Beineke, Derived graphs and digraphs. Beiträge zur Graphentheorie (H. Sachs, H. Voss, and H. Walther, eds.) Teubner, Leipzig 1968, pp 17-33.

        \bibitem{11Berm}
        J.C. Bermond, N. Homobono, C. Peyrat, Connectivity of Kautz networks, Disc. Math. 114 (1993), pp 51-62.
    \end{enumerate}
\end{frame}

\end{document} 
