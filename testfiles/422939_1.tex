% https://tex.stackexchange.com/a/422939
%\documentclass{beamer}
\documentclass[handout,t]{beamer}
%\usepackage{graphicx}
\usepackage{url}
\usepackage[brazil]{babel}  
\usepackage[utf8]{inputenc}
\usepackage{setspace}
\usepackage{mathtools}
\usepackage{wrapfig,lipsum}
\usepackage{ragged2e}
%\usepackage{etoolbox}
%\batchmode
\usepackage{amsmath,amssymb}
%\usepackage{enumerate}
\usepackage{epsfig,bbm,calc}
%\usepackage{color
\usepackage{ifthen,capt-of}
\usetheme{Berlin}
%\usecolortheme{senac}

%-------------------------Titulo/Autores/Orientador------------------------------------------------
\title[IGTDT Centre]{Activities}
\subtitle{Meeting}
\institute[]{\large IGTDT Centre \\ GYUHJUYT}
\date{27th June - 2017}
\author[ Meeting]{JACK katil}

%-------------------------Logo na parte de baixo do slide------------------------------------------
%\pgfdeclareimage[height=0.7cm]{iccsir_logo3}{iccsir_logo3}
%\logo{\pgfuseimage{iccsir_logo3}\hspace*{0.5cm}}

%-------------------------Este código faz o menuzinho bacana na parte superior do slide------------
\AtBeginSection[]
{
  \begin{frame}<beamer>
    \frametitle{Outline}
    \tableofcontents[currentsection]
  \end{frame}
}
\beamerdefaultoverlayspecification{<+->}
% -----------------------------------------------------------------------------
\begin{document}
% -----------------------------------------------------------------------------

%---Gerador de Sumário---------------------------------------------------------
\frame{\titlepage}
\section[]{}
\begin{frame}{Contents}
  \tableofcontents
\end{frame}
%---Fim do Sumário------------------------------------------------------------


% -----------------------------------------------------------------------------
\section{Past Years’ activities}
\begin{frame}{Past 5 Years’ activities}
\begin{itemize}
    \item  Dipole and sea salinity
    \item  rainfall over  region
    \item Satellite data of cloud –  climatology    
\end{itemize}
\end{frame}
%------------------------------------------------------------------------------

%------------------------------------------------------------------------------

\begin{frame}{Next Years’ activities}
\begin{itemize}
    \item change mitigation
    \item Study of ground water over 
    \item Building 
    \item Weather 
\end{itemize}
\end{frame}
%------------------------------------------------------------------------------
\section{SWOT}
\begin{frame}{SWOT Ana}
\begin{columns}
\begin{column}{1\textwidth}
    \includegraphics[width=\textwidth]{example-image}
\end{column}
\end{columns}
\end{frame}

%------------------------------------------------------------------------------
\section{El Niño-2017}
\begin{frame}{El Niño Condition}
\centering
\includegraphics[width=10cm, height=5cm]{example-image}
\begin{itemize}
    \item Normal conditions are present.

    \item Equatorial sea surface temperatures (SSTs) are near-to average in the central and east-central Pacific Ocean. 

\end{itemize}
\end{frame}
%------------------------------------------------------------------------------
\begin{frame}[c]{El Niño Condition}
\centering
\vspace{3pt}
\includegraphics[scale=0.4]{example-image}
\begin{itemize}
    \vspace{3pt}
    \item In absence of El Nino, the Southwest Monsoon 2017 is expected to be NORMAL.

\end{itemize}
\end{frame}
%------------------------------------------------------------------------------

%------------------------------------------------------------------------------
\section{Seasonal Monsoon Forecast}
\begin{frame}{Seasonal Monsoon Forecast-USA Model}
    \begin{columns}[onlytextwidth,T]
        \begin{column}{.54\textwidth}
            \includegraphics[width=\textwidth]{example-image}
        \end{column}
        \begin{column}{.47\textwidth}
            \begin{itemize}
                \item Part of normal rainfall
                \item During later possible.
            \end{itemize}
        \end{column}
    \end{columns}
\end{frame}
%------------------------------------------------------------------------------
\begin{frame}{Seasonal Monsoon}
\begin{columns}[onlytextwidth,T]
    \begin{column}{.6\textwidth}
        \includegraphics[width=\textwidth]{example-image}
    \end{column}
    \begin{column}{.43\textwidth}
        \begin{itemize}
            \item The rainfall during summer 
            \item Part of  normal rainfall.
        \end{itemize}
    \end{column}
\end{columns}
\end{frame}


%------------------------------------------------------------------------------

%------------------------------------------------------------------------------
\section{South America Rainfall}

    \begin{frame}{South America Rainfall }
    \begin{columns}[onlytextwidth,T]
        \begin{column}{.4\textwidth}
            \includegraphics[width=\textwidth]{example-image}
        \end{column}
        \begin{column}{0.6\textwidth}
            \begin{itemize}
                \addtobeamertemplate{block begin}{}{\justifying}
                \item\justifying Brazil : In their June Crop Report
                \item Sorghum and rice  practically unchanged.

            \end{itemize}
        \end{column}
    \end{columns}
\end{frame}


%------------------------------------------------------------------------------

\begin{frame}{South America (Brazil and Argentina) : Rainfall}
\begin{columns}[onlytextwidth,T]
    \begin{column}{.4\textwidth}
        \includegraphics[width=\textwidth]{example-image}
    \end{column}
    \begin{column}{0.6\textwidth}
        \begin{itemize}
            \addtobeamertemplate{block begin}{}{\justifying}
            \item\justifying Wheat production  more with a projected production of 31.5 million tons. 

        \end{itemize}
    \end{column}
\end{columns}
\end{frame}

%------------------------------------------------------------------------------

% -----------------------------------------------------------------------------
 \section{}
 \begin{frame}[c]
 \centering

 \textbf{\large Thank You for the attention}\\
   \vspace*{1cm}     
    Suggestions for providing additional services /informations are most welcome 

\end{frame}
% -----------------------------------------------------------------------------
\section{Monsoon Rainfall-2017}
\begin{frame}{Cumulative Rainfall as on 25th June, 2017}
\begin{columns}[onlytextwidth,T]
    \begin{column}{.45\textwidth}
        \includegraphics[width=\textwidth]{example-image}
    \end{column}
    \begin{column}{.47\textwidth}
        \begin{itemize}
            \item Country =  - 1 \% 
            \item Ka \& So region : Rainfall = - 52 \% 
            \item Kon \& region: Rai= - 3 \% 
            \item Marfall = + 47 \%
        \end{itemize}
    \end{column}
\end{columns}
\end{frame}


% -----------------------------------------------------------------------------
\begin{frame}{District wise Cumulative Rainfall}
\begin{columns}[onlytextwidth,T]
    \begin{column}{.45\textwidth}
        \includegraphics[width=\textwidth]{example-image}
    \end{column}
    \begin{column}{.48\textwidth}
        \begin{itemize}
            \item\justifying Centralred zones).

            \item\justifying The exd normal rainfall (Green and blue zones). 
        \end{itemize}
    \end{column}
\end{columns}
\end{frame}


%-----------------------------------------------------------------------------

\begin{frame}{District wise Cumulative Rainfall}
\begin{columns}[onlytextwidth,T]
    \begin{column}{.45\textwidth}
        \includegraphics[width=\textwidth]{example-image}
    \end{column}
    \begin{column}{.47\textwidth}
        \begin{itemize}
            \item\justifying Most part of the
            \vspace{3pt} 
            \item\justifying   some part ofrainfall (Red zones). 
        \end{itemize}
    \end{column}
\end{columns}
\end{frame}
\end{document}
-----------------------------------------------------------------------------

%-----------------------------------------------Este comentario nunca aparecera
