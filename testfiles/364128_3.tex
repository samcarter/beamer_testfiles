% https://tex.stackexchange.com/a/364128
\documentclass{beamer}

\usetheme{Madrid}

\title{LEoNIDS}

\subtitle{A Low-Latency and Energy-Efficient Network-Level Intrusion Detection System}

\author{NIKOS TSIKOUDIS \inst{1} ANTONIS PAPADOGIANNAKIS \inst{2} \and EVANGELOS P. MARKATOS \inst{2}}

\institute[NIT-KKR]{%
  \inst{1}
        Brandeis University, Waltham, MA 02453, USA
  \and \inst{2}
    Institute of Computer Science,Foundation for Research and Technology-Hellas, Heraklion 700 13, Greece
}

\date{IEEE Transaction on Emerging Topics in Computing \\ 26 Feburary 2016}
\subject{Cyber Security}

\begin{document}

\begin{frame}
  \titlepage
\end{frame}

\begin{frame}[allowframebreaks]{Contents}
    \tableofcontents[sections=1-4]
    \framebreak
    \tableofcontents[sections=5-10]
\end{frame}

\section{Introduction} 

\begin{frame}[allowframebreaks]{Introduction}
Low power consumption has emerged as one of the main
design goals in today's computer systems. Recently, much
effort has been put into improving the energy efficiency in
a variety of areas like data centers , high performance
computing , mobile devices , and networks .

Towards this direction, we aim to build an energy-efficient
Network-level Intrusion Detection System (NIDS). NIDS are
commonly deployed to detect security violations, enhancing
the secure operation of modern computer networks. They
perform computationally heavy operations like pattern
matching, regular expression matching, and other types of
complex analysis to detect at real time malicious activities in
the monitored network. Thus, NIDS usually utilize multi-core
systems or cluster of servers, to cope with increased
link speeds and complicated analysis.

\framebreak
However, the energy efficiency of security systems like NIDS has not received significant attention and has not been studied before.Although NIDS are usually provisioned to operate at link
rate, in order to be able to handle a fully utilized network,
most networks are typically much less utilized. This results
in increased power consumption at low traffic load. To reduce
the energy spent under low traffic we aim at building a power-
proportional NIDS using Dynamic Voltage and Frequency
Scaling (DVFS) and sleep states (C-states), which can be
found in modern processors. The system should consume
the less power needed to sustain the incoming traffic load.We found that a NIDS consumes less power when it uses the smallest number of cores that can operate at the lowest
possible frequency to process the network traffic, by keeping
these cores nearly fully utilized. This energy-efficient NIDS
can process all packets with up to 23 percent lower power
consumption than the original system at low rates. However,
we observe a significant increase on the detection latency due
to higher processing times when reducing the frequency, and
mostly due to increased queuing delays imposed by the high
utilization.

A low detection latency is very important to ensure a timely
reaction to the attack. Upon the detection of a packet that
carries an attack, the NIDS can actively terminate the offend-
ing connection or install a new firewall rule. This reaction
should be immediate, before the attack packets reach the
victim's machine and the attack succeeds. Therefore, our
results indicate a new tradeoff for NIDS: the energy-latency
tradeoff. Our key idea to resolve this tradeoff is to identify
the most important packets for attack detection and process
them with higher priority, resulting in low latency and fast
detection. The rest packets are processed with lower priority
to achieve an overall low power consumption.

We explore two alternative approaches to reduce the
latency of high-priority packets: time sharing and space
sharing. In time sharing we use a typical priority queue
scheduling in each core. In space sharing the high-priority
packets follow a different path, using dedicated cores with
much lower utilization to achieve low latency. To implement
space sharing we use features of modern network interface
cards (NIC) to move efficiently the processing of least-
significant packets to cores with higher utilization, a tech-
nique we call as flow migration. We experimentally compare
the two approaches and we find that space sharing has a better
power-latency ratio.

Based on these approaches we propose LEoNIDS: a NIDS
architecture that resolves the energy-latency tradeoff. The
implementation of LEoNIDS uses NIC features, a specialized
kernel module, a modified user-level library, and it is based on
the popular Snort NIDS [8]. LEoNIDS consumes less power,
proportionally to the traffic load, while its detection latency
remains low and almost constant at any traffic load.

The main contributions of this work are:

1. We identify a new tradeoff for NIDS: the energy-latency
tradeoff.

2. We resolve the energy-latency tradeoff.

3. We introduce space sharing.

4. We experimentally compare two alternative approaches for low latency in a power-proportional NIDS.

5.We present the design, implementation, and evaluation
of LEoNIDS.
\end{frame}

\section{Motivation}
\subsection {Why Detection Latency Matters}

\begin{frame}
  \frametitle{Motivation}    
\end{frame}

\subsection{Why Power Consumption Matters }
\section{Towards Power Proportional NIDS} 
\subsection{Experimental Environment}
\subsection{Power Consumption}
\subsection{Adapt to the Traffic Load}
\section{Energy-Latency Tradeoff in NIDS} 
\subsection{Detection Latency}
\subsection{Deconstructing Detection Latency}
\subsection{Delay Analysis}
\section{Solving the Energy-Latency Tradeoff} 
\subsection{Identify The Most Important Packets For Detection Latency}
\subsection{Tolerating Evasion Attempts}
\subsection{Time Sharing}
\subsection{Space Sharing}
\subsection{Delay Analysis With Priorities}
\section{Implementation} 
\subsection{Time Sharing}
\subsection{Space Sharing}
\section{Experimental Evaluation} 
\subsection{Comparing Time and Space Sharing}
\subsection{Comparing All Approaches}
\section{Related Work} 
\section{Conclusions} 

\section{Summary and Outlook} 

\begin{frame}{Summary}
  \begin{itemize}
  \item
    The \alert{first main message} of your talk in one or two lines.
  \item
    The \alert{second main message} of your talk in one or two lines.
  \item
    Perhaps a \alert{third message}, but not more than that.
  \end{itemize}

  \begin{itemize}
  \item
    Outlook
    \begin{itemize}
    \item
      Something you haven't solved.
    \item
      Something else you haven't solved.
    \end{itemize}
  \end{itemize}
\end{frame}

\section{References}  

\begin{frame}[allowframebreaks]
  \frametitle<presentation>{For Further Reading}

  \begin{thebibliography}{10}

  \beamertemplatebookbibitems
  % Start with overview books.

  \bibitem{Author1990}
    A.~Author.
    \newblock {\em Handbook of Everything}.
    \newblock Some Press, 1990.


  \beamertemplatearticlebibitems
  % Followed by interesting articles. Keep the list short. 

  \bibitem{Someone2000}
    S.~Someone.
    \newblock On this and that.
    \newblock {\em Journal of This and That}, 2(1):50--100,
    2000.
  \end{thebibliography}
\end{frame}

\end{document}
