% https://tex.stackexchange.com/a/341209
\documentclass{beamer}

\usetheme{Berlin}
\usepackage{booktabs} % Allows the use of \toprule, \midrule and \bottomrule in tables
\usepackage[english]{babel}
\usepackage{tikz}


\title[Essays on Urban and Environmental Economics]{\textbf{Essays on Urban and Environmental Economics}} %
\author[Nicola De Vivo]{Nicola De Vivo} % Your name
\institute[IMT]{
15th December 2016 \\ % Your institution for the title page
}

\date{
\begin{footnotesize}
%\textbf{IMT School for Advanced Studies Lucca}\\ \smallskip
%PhD Programme in Economics, Markets, Institutions - XXVII Cycle\\ \smallskip
%15th December 2016\\ \smallskip
\begin{flushleft}
\textbf{Advisor:} Bla Bla (Bla Bla)\\
\textbf{Co-Advisor:} Bla Bla (Bla Bla)
\end{flushleft}
\end{footnotesize}
}

\begin{document}

\begin{frame}
\vspace{-0.01cm}
\begin{center}
\tikz [remember picture, overlay]
    \node at
        (current page.north)
        {\includegraphics[trim=0cm 6.5cm 0cm 1.5cm, clip, width=.5\textwidth,height=.2\textheight]{example-image}};
\end{center}
\titlepage % Print the title page as the first slide
\end{frame}
%\end{document}

%\end{document}

\begin{frame}
\frametitle{Outline} % Table of contents slide, comment this block out to remove it
\tableofcontents % Throughout your presentation, if you choose to use \section{} and \subsection{} commands, these will automatically be printed on this slide as an overview of your presentation
\end{frame}

\section[(Bla Bla]{Bla Bla}
\setcounter{subsection}{1}

\begin{frame}
\frametitle{Main research questions} % Table of contents slide, comment this block out to remove it

%\begin{center}
%\textbf{Main research questions}
%\end{center}

\begin{itemize}
\item ""Bla Bla"" \pause
\item ""Bla Bla""
\end{itemize}

\end{frame}

\begin{frame}
\frametitle{Introduction}
\begin{center}
\textit{""Bla Bla""}\\ Fazio and Modica (2012)
\end{center}
\end{frame}

\begin{frame}
\frametitle{Introduction}
Population distribution is not random, it follows a certain distribution\\
Three possible candidates:\\
\begin{enumerate}
\item Bla Bla \pause
\item Bla Bla \pause
\item Bla Bla
\end{enumerate}
Puzzle caused by two empirical regularities:\\
\begin{enumerate}
\item Bla Bla \pause
\item Bla Bla
\end{enumerate}
\end{frame}

\begin{frame}
\frametitle{Bla Bla}
Formally established by Bla Bla (1949)\\
City sizes are said to satisfy a peculiar Bla Bla:\\

\begin{equation*}
Bla Bla
\end{equation*}

or, equivalently, the Bla Bla rule:

\begin{equation*}
Bla Bla
\end{equation*}

usually expressed in logarithmic terms:

\begin{equation*}
Bla Bla
\end{equation*}

\end{frame}

\begin{frame}
\frametitle{Bla Bla}
Formally established by Bla Bla (1931)\\
Growth rate of Bla Bla does not depend on the size of the Bla Bla (Bla Bla process)\\

\begin{figure}[h!]
\begin{center}
\includegraphics[trim=2cm 14cm 1.5cm 2.5cm, clip=true,width=8.0cm,height=4.5cm]{example-image}
\end{center}
\end{figure}

\end{frame}

\begin{frame}
\frametitle{Why a puzzle?}
Bla Bla is a Pareto distribution with exponent equal to 1\\
\begin{itemize}
\item Bla Bla (1999)
\item Bla Bla (1980)
\item Bla Bla (2005)
\end{itemize}
Bla Bla leads to a Log-Normal distribution\\
\begin{itemize}
\item Bla Bla (2004)
\item Bla Bla (1997)
\item Bla Bla (2003)
\end{itemize}
\end{frame}

\begin{frame}
\frametitle{Trying to solve the puzzle}
Bla Bla process plus ""something else""
\begin{itemize}
\item Bla Bla (1999)
\item Bla Bla (2000)
\end{itemize}
Double Bla Bla distribution
\begin{itemize}
\item Bla Bla (2002)
\item Bla Bla (2010)
\end{itemize}
\end{frame}

\begin{frame}
\frametitle{Our data}
Population of all Bla Bla Bla Bla on 3 censuses: 1991, 2001 e 2011\\
\begin{center}
\textbf{Why?}
\end{center}
\begin{itemize}
\item No unique definition of Bla Bla  \pause
\item All definitions given for statistical purposes \pause
\item Bla Bla do not cover all Bla Bla \pause
\item Proper Bla Bla definition based only on Bla Bla criteria
\end{itemize}
\end{frame}

\begin{frame}
\frametitle{Results}
\begin{itemize}
\item ""What is the actual Bla Bla of Bla Bla?""
\end{itemize}
\begin{figure}[h!]
\begin{center}
\includegraphics[trim=0.5cm 7.5cm 0.0cm 7.5cm, clip=true, width=8.0cm,height=5cm]{example-image}
\end{center}
\begin{center}
Not Bla Bla
\end{center}
\end{figure}
\end{frame}

\begin{frame}
\frametitle{Results}
\begin{itemize}
\item ""What is the actual Bla Bla of Bla Bla?""
\end{itemize}
\begin{figure}[h]
\begin{center}
\begin{tabular}{cc}
\includegraphics[trim=1.5cm 7cm 0.5cm 7cm, clip=true,width=0.5\textwidth,height=4.5cm]{example-image} &
\includegraphics[trim=3cm 9cm 3cm 9cm, clip=true,width=0.5\textwidth,height=4.55cm]{example-image}
\end{tabular}
\end{center}
\end{figure}
\begin{center}
Not Bla Bla
\end{center}
\end{frame}

\begin{frame}
\frametitle{Results}
\begin{itemize}
\item ""What is the actual Bla Bla of Bla Bla?""
\end{itemize}
\begin{figure}[h]
\begin{center}
\begin{tabular}{cc}
\includegraphics[trim=3cm 9cm 3cm 9cm, clip=true,width=0.5\textwidth,height=4.5cm]{example-image} &
\includegraphics[trim=3cm 9cm 3cm 9cm, clip=true,width=0.5\textwidth,height=4.5cm]{example-image}
\end{tabular}
\end{center}
\end{figure}
\begin{center}
Double Bla Bla!
\end{center}
\end{frame}

\begin{frame}
\frametitle{The model}
\begin{equation*}
dX=\mu Xdt+\sigma Xd\omega
\end{equation*}
Probability of creating new Bla Bla: $\lambda$ $dt$ in ($t$,$t+dt$)
\begin{center}
$\Downarrow$
\end{center}
\begin{equation*}
\begin{split}
f(x)& =\frac{\alpha\beta}{\alpha+\beta} \left[x^{-\alpha-1}\exp\left\{\alpha\mu_{0}+\frac{\alpha^{2}\sigma_{0}^{2}}{2}\right\}
      \Phi\left(\frac{\ln(x)-\mu_{0}-\alpha\sigma_{0}^{2}}{\sigma_{0}}\right)\right. +\\
    & \quad \left.x^{\beta-1}\exp\left\{-\beta\mu_{0}+\frac{\beta^{2}\sigma_{0}^{2}}{2}\right\} \Phi^{c}\left(\frac{\ln(x)-\mu_{0}-\beta\sigma_{0}^{2}}{\sigma_{0}}\right)\right]
\end{split}
\end{equation*}
\end{frame}

\begin{frame}
\frametitle{Results of simulations}
\begin{figure}
\begin{center}
\includegraphics[trim=3.75cm 10.0cm 4cm 10cm, clip=true, width=10.0cm,height=6.5cm]{example-image}
\end{center}
\end{figure}
\end{frame}

\begin{frame}
\frametitle{A change in the paradigm}
\begin{itemize}
\item Does Bla Bla holds for Bla Bla?
\end{itemize}
\begin{figure}
\begin{center}
\begin{tabular}{cc}
\includegraphics[trim=3.5cm 9.0cm 3.5cm 9.5cm, clip=true,width=0.5\textwidth,height=5cm]{example-image} &
\includegraphics[trim=3.5cm 9.0cm 3.5cm 9.5cm, clip=true,width=0.5\textwidth,height=5cm]{example-image}
\end{tabular}
\end{center}
\end{figure}
\end{frame}

\begin{frame}
\frametitle{Conclusions}
\begin{itemize}
\item The way in which Bla Bla distribute across Bla Bla is still an open question \pause
\item Two widely recognized empirical regularities: Bla Bla and Bla Bla \pause
\item Double Bla Bla seems to be the most suitable for Bla Bla \pause
\item Model leading to it does not give good theoretical overlap \pause
\item Maybe a change in the Bla Bla is needed
\end{itemize}
\end{frame}

\section[Bla Bla]{Bla Bla: Testing Bla Bla for Bla Bla}
\setcounter{subsection}{1}

\begin{frame}
\frametitle{Introduction}
\begin{itemize}
\item Bla Bla (1999): theoretical model leading to a Bla Bla process that follows Bla Bla driven by Bla Bla
\item Bla Bla (2004): Bla Bla process characterized by two driving forces: Bla Bla process of local Bla Bla and perfect Bla Bla of Bla Bla
\end{itemize}
\begin{center}
$\Downarrow$
\end{center}
\begin{center}
Bla Bla characteristics and Bla Bla concur in the Bla Bla of Bla Bla and Bla Bla
\end{center}
\end{frame}

\begin{frame}
\frametitle{Introduction}
\begin{center}
Idiosyncratic reasons why Bla Bla decide to localize in a given Bla Bla or choose to move
\end{center}
\begin{center}
$\Downarrow$
\end{center}
\begin{itemize}
\item Bla Bla (2001): Young Bla Bla prefer working in larger Bla Bla
\item Bla Bla (2006): Different Bla Bla between old and young Bla Bla
\item Bla Bla (1998): Bla Bla migrate in densely Bla Bla areas, while Bla Bla prefer Bla Bla
\item Bla Bla (2015): \textsl{""as Bla Bla continues to Bla Bla, the Bla Bla of its Bla Bla has moved away from a Bla Bla and has been reshaped""}
\end{itemize}
\end{frame}

\begin{frame}
\frametitle{Main research questions}
\begin{itemize}
\item What is the level of Bla Bla? \pause
\item What is Bla Bla for Bla Bla? \pause
\item Are there any differences between these two variables? \pause
\item Are there any differences if we differentiate for Bla Bla?
\end{itemize}
\end{frame}

\begin{frame}
\frametitle{The data}
\begin{itemize}
\item Dataset provided by Bla Bla, Bla Bla Institute for Bla Bla Research
\item Annual observations for all Bla Bla, from 2001 to 2011
\item Total Bla Bla, divided by 5-year Bla Bla
\end{itemize}
\end{frame}

\begin{frame}
\frametitle{Bla Bla Analysis}
Method proposed by Bla Bla (2013): estimate Bla Bla point as a parameter of a Bla Bla distribution, $h(\cdot)$, by means of Bla Bla estimation:\\

\begin{equation*}
\begin{aligned}
& \underset{Bla Bla,Bla Bla,Bla Bla,q}{max}
& & \ln Bla Bla(P;Bla Bla,Bla Bla,Bla Bla,q),\\
& s.t. & & Bla Bla > \exp(Bla Bla)
\end{aligned}
\end{equation*}

Bla Bla coefficients in the Bla Bla rule estimated by a method proposed by Bla Bla (2011):

\begin{equation*}
\log(Bla Bla) = \log(Bla Bla)-q\log(Bla Bla)
\end{equation*}
\end{frame}

\begin{frame}
\frametitle{Bla Bla Analysis}
\begin{center}
Bla Bla coefficient vs. Bla Bla coefficient
\end{center}
\begin{figure}
\begin{center}
\begin{tabular}{cc}
\includegraphics[trim=0cm 5.5cm 0cm 5.5cm, clip=true,width=0.5\textwidth,height=5cm]{example-image} &
\includegraphics[trim=0cm 5.5cm 0cm 5.5cm, clip=true,width=0.5\textwidth,height=5cm]{example-image}
\end{tabular}
\end{center}
\end{figure}
\end{frame}

\begin{frame}
\frametitle{Bla Bla Analysis}
\begin{center}
Bla Bla rate vs. Bla Bla rate
\end{center}
\begin{figure}
\begin{center}
\begin{tabular}{cc}
\includegraphics[trim=0cm 2cm 0cm 2cm, clip=true,width=0.5\textwidth,height=5cm]{example-image} &
\includegraphics[trim=0cm 2cm 0cm 2cm, clip=true,width=0.5\textwidth,height=5cm]{example-image}
\end{tabular}
\end{center}
\end{figure}
\end{frame}

\begin{frame}
\frametitle{Bla Bla Analysis}
\begin{center}
Bla Bla rate vs. Bla Bla rate
\end{center}
\begin{figure}
\begin{center}
\begin{tabular}{cc}
\includegraphics[trim=0cm 2cm 0cm 2cm, clip=true,width=0.5\textwidth,height=5cm]{example-image} &
\includegraphics[trim=0cm 2cm 0cm 2cm, clip=true,width=0.5\textwidth,height=5cm]{example-image}
\end{tabular}
\end{center}
\end{figure}
\end{frame}

\begin{frame}
\frametitle{Bla Bla Analysis}
\begin{center}
Bla Bla rate vs. Bla Bla rate
\end{center}
\begin{figure}
\begin{center}
\begin{tabular}{cc}
\includegraphics[trim=0cm 2cm 0cm 2cm, clip=true,width=0.5\textwidth,height=5cm]{example-image} &
\includegraphics[trim=0cm 2cm 0cm 2cm, clip=true,width=0.5\textwidth,height=5cm]{example-image}
\end{tabular}
\end{center}
\end{figure}
\end{frame}

\begin{frame}
\frametitle{Conclusions}
\begin{itemize}
\item First attempt to introduce different Bla Bla variables and Bla Bla in Bla Bla analysis \pause
\item Bla Bla much more concentrated than Bla Bla \pause
\item Bla Bla is in operation for Bla Bla but not for Bla Bla \pause
\item Bla Bla tend to agglomerate in Bla Bla, while Bla Bla show the opposite behavior
\end{itemize}
\end{frame}

\section[Neutrality of Bla Bla]{How neutral is the choice of the Bla Bla in Bla Bla schemes? Evidence from the Bla Bla}
\setcounter{subsection}{1}

\begin{frame}
\frametitle{The Bla Bla}
\begin{itemize}
\item European Union Bla Bla Scheme
\item Main initiative of the European Union to reach Kyoto targets
\item Cap-and-trade scheme: emissions permits are exogenously capped and then allocated to participants
\item Three different periods: 2005-2007, 2008-2012, 2013-2020
\item Penalty for not complying
\end{itemize}
\end{frame}

\begin{frame}
\frametitle{The EU ETS}
\begin{itemize}
\item Covers all Bla Bla
\item Covers about Bla Bla
\item Accounts for Bla Bla
\item Unilaterally introduced $\Rightarrow$ Bla Bla
\item Three exemption criteria:
\begin{enumerate}
\item Bla Bla
\item Bla Bla criterion
\item Bla Bla criterion
\end{enumerate}
\end{itemize}
\end{frame}

\begin{frame}
\frametitle{Theoretical Bla Bla}
\begin{itemize}
\item Free Bla Bla can have distortionary effects
\item Absence of distortionary effects necessary to have effectiveness in Bla Bla scheme
\item Bla Bla (1960): in an ideal world, Bla Bla are independent
\item In real world Bla Bla schemes, free Bla Bla distorts Bla Bla outcomes:
\begin{itemize}
\item Bla Bla (1995): in presence of Bla Bla costs
\item Bla Bla \textit{et al.} (1991): in presence of Bla Bla anomalies
\end{itemize}
\end{itemize}
\end{frame}

\begin{frame}
\frametitle{Theoretical Bla Bla}
Evaluate causal relationship between Bla Bla and Bla Bla still challenging\\
Only two papers
\begin{enumerate}
\item Bla Bla (2013):
\begin{itemize}
\item Focus on Bla Bla
\item Bla Bla variable approach
\item Not significant Bla Bla effect \pause
\end{itemize}
\item Bla Bla and Bla Bla (2008):
\begin{itemize}
\item Spanish Bla Bla plants during Bla Bla 
\item Non Bla Bla in Bla Bla allocation rule
\item Not significant Bla Bla effect
\end{itemize}
\end{enumerate}
\end{frame}

\begin{frame}
\frametitle{Results}
\begin{center}
Bla Bla trends
\end{center}
\begin{figure}
\centering
\begin{tabular}{cc}
\includegraphics[trim=4cm 10cm 4cm 10cm, clip=true,width=0.5\textwidth,height=5cm]{example-image} &
\includegraphics[trim=0cm 8cm 0cm 7.5cm, clip=true,width=0.5\textwidth,height=5cm]{example-image}
\end{tabular}
\end{figure}
\end{frame}

\begin{frame}
\frametitle{Results}
\begin{center}
Bla Bla rate comparison
\end{center}
\begin{figure}
\centering
\begin{tabular}{cc}
\includegraphics[trim=3cm 9cm 3cm 10cm, clip=true,width=0.5\textwidth,height=5.0cm]{example-image} &
\includegraphics[trim=3cm 9cm 3cm 10cm, clip=true,width=0.5\textwidth,height=5.0cm]{example-image}
\end{tabular}
\end{figure}
\end{frame}

\begin{frame}
\frametitle{Results}
\begin{center}
Bla Bla rate comparison
\end{center}
\begin{figure}
\centering
\begin{tabular}{cc}
\includegraphics[trim=3cm 9cm 3cm 10cm, clip=true,width=0.5\textwidth,height=5.0cm]{example-image} &
\includegraphics[trim=3cm 9cm 3cm 10cm, clip=true,width=0.5\textwidth,height=5.0cm]{example-image}
\end{tabular}
\end{figure}
\end{frame}

\begin{frame}
\frametitle{Bla Bla analysis}
\begin{itemize}
\item Focus on Bla Bla plants
\item Bla Bla approach
\item Estimate of the equation:
\begin{equation*}
log(Bla Bla_{it})=\beta Bla Bla_{s}\times Post2013_{t}+ X'\gamma + \tau_{t} \alpha_{i}+\varepsilon_{it}
\end{equation*}
\item Interest in $\beta$
\item Assignment to Bla Bla not random $\Rightarrow$ Bla Bla Bla Bla to Bla Bla
\end{itemize}
\end{frame}

\begin{frame}
\frametitle{Bla Bla analysis}
Bla Bla (simple Bla Bla on Bla Bla)
\begin{table}
\centering
\tiny
\label{tab:att}
\begin{tabular}{lcccc}
\toprule
Bla Bla &  & Bla Bla  & SE & t-test\\
\midrule
Change in Bla Bla  & Unmatched & -0.0437 & (0.0220) & -1.99\\
 & Bla Bla & 0.0255 & (0.0292) & 0.87\\
Change in Bla Bla  & Unmatched & -0.0119 & (0.0240) & -0.49\\
 & Bla Bla & 0.0496 & (0.0318) & 1.56\\
Change in Bla Bla & Unmatched & 0.0094 & (0.0309) & 0.30\\
 & Bla Bla & 0.0754 & (0.0394) & 1.91*\\
Change in Bla Bla & Unmatched & -0.0232 & (0.0331) & -0.70\\
 & Bla Bla & 0.0552 & (0.0407) & 1.36\\
Change in Bla Bla & Unmatched & 0.0419 & (0.0424) & 0.99\\
 & Bla Bla & 0.1365 & (0.0453) & 3.01***\\
Change in Bla Bla & Unmatched & 0.0667 & (0.0482) & 1.38\\
 & Bla Bla & 0.1908 & (0.0556) & 3.43***\\
\bottomrule
\end{tabular}
\end{table}
\end{frame}

\begin{frame}
\frametitle{Bla Bla Analysis}
Bla Bla with Bla Bla
\begin{table}
\centering
\tiny
\label{tab:diff_in_diff}
\begin{tabular}{lcccccc}
\toprule
Bla Bla & (1) & (2) & (3) & (4) & (5) & (6)\\
\midrule
Bla Bla & 0.0133 & 0.0473* & 0.0445* & 0.0431* & 0.0686** & 0.0438\\
 & (0.0211) & (0.0270) & (0.0269) & (0.0228) & (0.0281) & (0.0277)\\
Bla Bla & 0.0762** & 0.151*** & 0.168*** & 0.136*** & 0.194*** & 0.169***\\
 & (0.0353) & (0.0453) & (0.0451) & (0.0383) & (0.0441) & (0.0437)\\
Bla Bla &  &  & 0.509*** &  &  & 0.567***\\
 &  &  & (0.0753) &  &  & (0.0957)\\
 \midrule
Bla Bla & No & Yes & Yes & No & Yes & Yes\\
Bla Bla & Yes & Yes & Yes & Yes & Yes & Yes\\
Bla Bla & No & No & No & Yes & Yes & Yes\\
Bla Bla & No & No & No & Yes & Yes & Yes\\
\midrule
N & 20125 & 19810 & 19810 & 20125 & 19810 & 19810\\
\bottomrule
\end{tabular}
\end{table}
\end{frame}

\begin{frame}
\frametitle{Bla Bla Analysis}
Effect for different Bla Bla criteria
\begin{table}
\centering
\tiny
\label{tab:diff_in_diff_criteria}
\begin{tabular}{lcc}
\toprule
log(verified emissions) & (1) & (2)\\
\midrule
Bla Bla & 0.0632*** & 0.0660** \\
 & (0.0239) & (0.0287)   \\
Bla Bla & 0.222*** & 0.227***\\
 & (0.0418) & (0.0476)   \\
Bla Bla & -0.00743 & -0.340***\\
 & (0.0304) & (0.0638)   \\
Bla Bla & -0.0324 & -0.697***\\
 & (0.0625) & (0.126)   \\
Bla Bla & 0.0619*** & 0.0577** \\
 & (0.0236) & (0.0266)   \\
Bla Bla & 0.150*** & 0.142***\\
 & (0.0423) & (0.0508)   \\
Bla Bla & -0.0642* & -0.00911   \\
 & (0.0349) & (0.0370)   \\
Bla Bla & -0.0443 & 0.0659   \\
 & (0.0603) & (0.0590)   \\
\midrule
Bla Bla & Yes & Yes\\
Bla Bla & Yes & Yes\\
Bla Bla & No & Yes\\
Bla Bla & No & Yes\\
\midrule
N & 19810 & 19810\\
\bottomrule
\end{tabular}
\end{table}
\end{frame}

\begin{frame}
\frametitle{Conclusions}
\begin{itemize}
\item Empirical evaluation of the Bla Bla in Bla Bla schemes \pause
\item Based on data on Bla Bla \pause
\item Focus on Bla Bla establishments \pause
\item Our estimate suggests an increase in Bla Bla for Bla Bla that are exempted from Bla Bla \pause
\item Contradiction with Bla Bla prediction \pause
\item Bla Bla schemes efficiency could be improved
\end{itemize}
\end{frame}

\begin{frame}
\frametitle{Acknowledgement}
\begin{itemize}
\item \textbf{Advisor}: Bla Bla
\item \textbf{Co-Advisor}: Bla Bla
\item \textbf{Italian cities}: Bla Bla
\item \textbf{Ageing and Labor Market}: Bla Bla, Bla Bla and Bla Bla
\item \textbf{Allocation mechanism}: Bla Bla
\end{itemize}
\end{frame}

\begin{frame}
\frametitle{Now, let's go for a beer}
\begin{figure}[h]
\begin{center}
\includegraphics[width=0.8\textwidth,height=0.75\textheight]{example-image}
\end{center}
\end{figure}
\end{frame}

\end{document}
