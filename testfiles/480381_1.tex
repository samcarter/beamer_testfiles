% https://tex.stackexchange.com/a/480381
\documentclass[aspectratio=169, xcolor={x11names}]{beamer}

\usecolortheme{wolverine}
\useoutertheme[]{split}
\useinnertheme{inmargin}

\setbeameroption{show notes}

\setbeamersize{description width=0mm}                       
%\renewcommand{\encodingdefault}{T1}

\usepackage{cleveref}
\crefformat{equation}{Eq. (#1)}

\begin{document}

\begin{frame}<-2>[label=foo]

    \begin{block}{Derivation of Equivalent Impedance}

        \begin{description}[<+->]
            \item[Item 1] Item 1
            \item[Item 2] Item 2
            \item[Item 3] Item 3
                \begin{equation}e = m c^{2}\label{eqx}\end{equation}\cref{eqx} is what we want.
        \end{description}

    \end{block}\end{frame}

 \note{\begin{equation}e = m c^{2}\label{eq}\end{equation}\cref{eq} is what we want.}

 \againframe<3->{foo}    

\end{document}
