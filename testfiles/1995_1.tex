% https://topanswers.xyz/tex?q=1769#a1995
\documentclass{beamer}

\usepackage{tikz,tcolorbox}
\usetikzlibrary{shapes,calc}
\usepackage{fontawesome}
\tcbuselibrary{skins}
\usecolortheme{crane}

  
  \newtcolorbox{important}[1][]{
    before upper={\setbeamercolor{item}{fg=red}},
  	enhanced,
  	arc=5mm,
  	drop lifted shadow=red!50,
  	colback=red!5,
  	colframe=white,
  	leftrule=0mm,%
  	detach title,
  	overlay unbroken and first ={
  	\node[red,anchor=north east,scale=1.3] 
  	at (frame.north west) {\faInfoCircle};
  	}        
  }


\begin{document}

\begin{frame}

\begin{important}
L'assurance vie a en réalité deux principes
	\begin{enumerate}
		\item quand elle est une assurance
		      \begin{itemize}
		      	\item son but est de constituer une épargne générant des intérêts, et servant au moment du dénouement, la retraite par exemple, à financer une rente viagère ou à percevoir un capital pour un projet
		      \end{itemize}
		\item quand elle est une assurance décès
		      \begin{itemize}
		      	\item elle permet de léguer une partie de son patrimoine à un bénéficiaire désigné dans le contrat.   	      
		      \end{itemize}				      		
	\end{enumerate}
\end{important}	 

\end{frame}

\end{document}