% https://tex.stackexchange.com/a/413350
\documentclass[pdf]{beamer}

\usetheme{Malmoe}  %% Themenwahl
\usecolortheme{beaver}
\usecolortheme{orchid}
\usefonttheme{professionalfonts}

\begin{document}

\begin{frame}{Frenet equations}

\begin{alertblock}{\textbf{Theorem:} Frenet equations}
    For the Frenet frame $(\bf t,\bf n)$ of a curve holds
    \begin{align*}
        \bf t'(t) &= \phantom+\kappa(t)\bf n(t),\\
        \bf n'(t) &= -\kappa(t)\bf t(t).
    \end{align*}%

    \vspace{-1em}
    {\usebeamercolor[bg]{block title alerted}\hrulefill}

    The equations can be written in \emph{matrix form}:
    \[\begin{pmatrix}
    \bf t'\\\bf n'
    \end{pmatrix}=\begin{pmatrix}
    \phantom-0&\kappa\\-\kappa&0
    \end{pmatrix}\begin{pmatrix}
    \bf t\\\bf n
    \end{pmatrix}.\]
\end{alertblock}

\end{frame}

\end{document}
