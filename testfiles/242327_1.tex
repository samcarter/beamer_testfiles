% https://tex.stackexchange.com/a/242327
\documentclass[11pt]{beamer}
\usetheme{Warsaw}
\usepackage[utf8]{inputenc}
\usepackage{amsmath,adjustbox,mathtools}
\usepackage{amsfonts}
\usepackage{amssymb}
\usepackage{graphicx}

\begin{document}

    \begin{frame}
        \frametitle{LU Factorization of A}
        \begin{equation}
        \resizebox{\textwidth}{!}{$\displaystyle
        \mathbf{LU} = \begin{pmatrix} \nonumber
        \gamma_1   \\
        -1  & \gamma_2  \\
        &  -1  & \gamma_3  \\
        &        & \ddots & \ddots \\
        &        &        &  -1 & \gamma_{N-2}  \\
        &        &        &          & -1 & \gamma_{N-1}  \\    
        \end{pmatrix}
        \begin{pmatrix}
        1  & \delta_1            \\
        & 1        & \delta_2 \\
        &          & 1        & \delta_3 \\
        &          &          &  \ddots & \ddots \\
        &          &          &         & 1     & \delta_{N-2} \\
        &          &          &         &       &  1
        \end{pmatrix}
        $}
        \end{equation}
        which is represented as
        \begin{align} \label{eqn:nonlinear2term}
        \gamma_1 &= 2+r  \nonumber \\ 
        \gamma_i &= 2+r-1/\gamma_{i-1},  \hspace{2mm} i=2, \dots ,N-1.
        \end{align}
    \end{frame}

\end{document}
