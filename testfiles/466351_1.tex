% https://tex.stackexchange.com/a/466351
\documentclass[xcolor=dvipsnames]{beamer}
\mode<presentation>
\usetheme{Madrid}


\makeatletter
\setbeamertemplate{theorem begin}
{%
    \setbeamercolor{block title}{bg=green!40!black}
  \begin{\inserttheoremblockenv}
  {%
    \textcolor{orange}{\inserttheoremname}
    \ifx\inserttheoremaddition\@empty\else\ \inserttheoremaddition\fi%
  }%
}
\makeatother

\begin{document}
\begin{frame}{}
\begin{theorem}[\cite{MS92}]
For a $d$-tuple of ....
\end{theorem}

\beamertemplatearticlebibitems
\begin{thebibliography}{9}
\bibitem[V. M\""{u}ller and A. Soltysiak (1992)]{MS92}
\alert{V. M\""{u}ller, and A. Soltysiak}
\newblock Spectral radius formula for commuting Hilbert space operators
\newblock \em Studia Math.103. (1992), 329-333.
\end{thebibliography}

\end{frame}
\end{document}
