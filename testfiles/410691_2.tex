% https://tex.stackexchange.com/a/410691
\RequirePackage{xr-hyper}
\documentclass{beamer}
\usepackage[utf8]{inputenc}    

\theoremstyle{plain}
\newtheorem{thm}{Sætning}[section]

\externaldocument{nameofyourthesisfile}

\begin{document}

\section{Præsentation af sætningen.}
\begin{frame}
\frametitle{Picard's sætning}

\begin{thm}[\ref{thm:second}]
Lad $G$ være åben i $\mathbb{C}$. En funktion $h:G\to \mathbb{R}$ kaldes 
harmonisk hvis $h\in C^2(G)$ og $\Delta h=0$ i $G.$ 
\end{thm}

\end{frame}
\end{document}
