% https://topanswers.xyz/tex?q=1767#a1992
\documentclass{beamer}

\usepackage{tikz,tcolorbox}
\usetikzlibrary{shapes,calc}
\usepackage{fontawesome}
\tcbuselibrary{skins}
\usecolortheme{crane}

\newcommand{\inserticon}[1]{%
  \ifstrequal{#1}{formula}{\faFlask}{}%
  \ifstrequal{#1}{net}{\faLink}{}%
  \ifstrequal{#1}{question}{\faQuestionCircle}{}%
  \ifstrequal{#1}{idea}{\faLightbulbO}{}%
  \ifstrequal{#1}{information}{\faInfoCircle}{}%    
}

  \newtcolorbox{information}[1][information]{
  	enhanced,
  	arc=5mm,
  	drop lifted shadow=blue!50,
  	colback=blue!5,
  	colframe=white,
  	leftrule=0mm,%
  	detach title,
  	overlay unbroken and first ={
  	\node[blue,anchor=north east,scale=1.3] 
  	at (frame.north west) {\inserticon{#1}};
  	}        
  }
 
  \newtcolorbox{important}[1][information]{
  	enhanced,
  	arc=5mm,
  	drop lifted shadow=red!50,
  	colback=red!5,
  	colframe=white,
  	leftrule=0mm,%
  	detach title,
  	overlay unbroken and first ={
  	\node[red,anchor=north east,scale=1.3] 
  	at (frame.north west) {\inserticon{#1}};
  	}        
  }

\begin{document}

\begin{frame}

	\begin{important}[information]
  \textit{If $_{\chi PP}(M)=2$ a haplotype matrix M we can find an optimal pp-partition in polynomial time}
	\end{important}  

	\begin{information}[formula]
  \textit{If $_{\chi PP}(M)=2$ a haplotype matrix M we can find an optimal pp-partition in polynomial time}
	\end{information}
  
 	\begin{information}[net]
  \textit{If $_{\chi PP}(M)=2$ a haplotype matrix M we can find an optimal pp-partition in polynomial time}
 	\end{information}
\end{frame}


\end{document}