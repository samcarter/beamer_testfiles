% https://tex.stackexchange.com/a/394217
\documentclass{beamer}
\mode<presentation> {

    %\usepackage{graphicx} 
    \usepackage{booktabs} 
    \usepackage{beamerthemesplit}
    \setbeamertemplate{footline}[frame number]

}

\begin{document}


\begin{frame}
    \begin{itemize}
        \item The usual necessary conditions for an optimum:
        \begin{align}
            F_{K}(K_{t}^{d},L_{t}^{d})&=r_{t}\\
            F_{L}(K_{t}^{d},L_{t}^{d})&=w_{t}
        \end{align}
        \item[-] (These two equations do not determine $K_{t}^{d}$ and $L_{t}^{d}$ from given $r_{t}$ and $w_{t}$; they only determine $\dfrac{K_{t}^{d}}{L_{t}^{d}}$
        \item In equilibrium:
        \begin{itemize} 
            \item The marginal product of capital equal the rental price of capital.
            \item The marginal product of labor equals the wage.
        \end{itemize}
%\begin{itemize}
        \item (Note that $F_{K}(K_{t},L_{t}$) represents the first derivative of the production function w.r.t. capital).
    \end{itemize}
\end{frame}

\end{document}
