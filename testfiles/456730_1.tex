% https://tex.stackexchange.com/a/456730
\documentclass[fleqn,9pt]{beamer}
\usetheme[
%%% options passed to the outer theme
%    hidetitle,           % hide the (short) title in the sidebar
%    hideauthor,          % hide the (short) author in the sidebar
%    hideinstitute,       % hide the (short) institute in the bottom of the sidebar
%    shownavsym,          % show the navigation symbols
%    width=2cm,           % width of the sidebar (default is 2 cm)
%    hideothersubsections,% hide all subsections but the subsections in the current section
%    hideallsubsections,  % hide all subsections
%    left                % right of left position of sidebar (default is right)
  ]{Aalborg}

\usepackage[utf8]{inputenc}
\usepackage{multicol}
\usepackage[english]{babel}
\usepackage[T1]{fontenc}
\usepackage{booktabs}
\usepackage[]{moresize}
\usepackage{lmodern} %optional
\usepackage{epstopdf}
\usepackage{appendixnumberbeamer}
\usepackage{amsmath,amsfonts,amsthm}
\usepackage{caption}
\usepackage{relsize}
\usepackage{mathtools}
\usepackage{multirow}
\DeclarePairedDelimiter{\ceil}{\lceil}{\rceil}
\newcommand\Fontvi{\fontsize{4.4}{7.2}\selectfont}
\newcommand\Fontt{\fontsize{5.5}{7.2}\selectfont}
\newcommand\Fonttt{\fontsize{6}{7.2}\selectfont}

\usepackage[absolute,overlay]{textpos}
\usepackage[noend]{algpseudocode}
\makeatletter
\algnewcommand{\LineComment}[1]{\Statex \hskip\ALG@thistlm \(\) #1}
\makeatother


\usepackage{listings}
\definecolor{dkgreen}{rgb}{0,0.6,0}
\definecolor{gray}{rgb}{0.5,0.5,0.5}
\definecolor{mauve}{rgb}{0.58,0,0.82}
\usepackage{icomma}

\usepackage{tcolorbox}
\usepackage{graphicx}
\theoremstyle{definition}
\newtheorem{prop}{Proposition}
%\newtheorem{theorem}{Theorem}[section]
%\newtheorem{lemma}[theorem]{Lemma}
% colored hyperlinks
\newcommand{\chref}[2]{%
  \href{#1}{{\usebeamercolor[bg]{Aalborg}#2}}
}
\usepackage{tikz}
\usepackage{forest}[compat=all]
\usetikzlibrary{arrows,shapes,positioning,calc,chains,fit,trees,decorations}
\definecolor{darblue}{rgb}{0.00,0.41,0.66}% dark blue
  \definecolor{liblue}{rgb}{0.55,0.79,0.94}% light blue


\usepackage{pgfplots}
\usepgfplotslibrary{fillbetween}
\pgfplotsset{compat=1.16}
\usetikzlibrary{patterns}
\usepackage{ragged2e}

\tikzset{
    decision/.style={rectangle, minimum height=10pt, minimum width=10pt, draw=black, fill=black!30!white, thin, inner sep=0pt},
    chance/.style={circle, minimum width=10pt, draw=black, fill=black!30!white, thin, inner sep=0pt},
    leaf-chance/.style={isosceles triangle, minimum width=10pt, draw=black, thin, fill=white, inner sep=0pt, shape border rotate=180, outer sep=-\pgflinewidth}
  }
\tikzset{
    invisible/.style={opacity=0},
    visible on/.style={alt={#1{}{invisible}}},
    alt/.code args={<#1>#2#3}{%
      \alt<#1>{\pgfkeysalso{#2}}{\pgfkeysalso{#3}} % \pgfkeysalso doesn't change the path
    },
  }
 \pgfmathdeclarefunction{poiss}{1}{%
  \pgfmathparse{(#1^x)*exp(-#1)/(x!)}%
  }

\tikzset{
myshape/.style={
  rectangle split,
  minimum height=1.5cm,
  rectangle split horizontal,
  rectangle split parts=5, 
  draw, 
  anchor=center,
  },
mytri/.style={
  draw,
  shape=isosceles triangle,
  isosceles triangle apex angle=60,
  inner xsep=0.65cm
  }
}

\begin{document}

\begin{frame}[fragile]{Introduction to Decision Trees}{Decision Trees}
\scriptsize
We can represent decision problems in a graphical form as {\color{red} \textbf{Decision Trees}}\\[1mm]
For our hotel situation we have:\\[5mm]

\definecolor{dgreen}{RGB}{0,102,51}

\begin{center}
\scalebox{0.75}{

\put(17,210){{\color{blue}\textbf{Land purchase decision}}}
\put(120,210){{\color{blue}\textbf{Airport location}}}
\put(200,210){{\color{blue}\textbf{Payoff}}}

\put(7,186){Buy A}
\put(7,131){Buy B}
\put(7,76){Buy A \& B}
\put(7,21){Buy nothing}

\put(10,175){-18}
\put(10,120){-12}
\put(10,65){-30}
\put(10,10){0}

\put(109,197.5){A}
\put(109,170){B}
\put(109,143){A}
\put(109,115.5){B}
\put(109,88){A}
\put(109,61){B}
\put(109,33){A}
\put(109,6.5){B}

\put(109,189.5){31}
\put(109.5,161){6}
\put(109.5,135){4}
\put(109,107.5){23}
\put(109,80){35}
\put(109,53){29}
\put(109.5,26){0}
\put(109.5,0){0}


\hspace*{-3.6cm}\begin{forest}
   my label/.style={
      edge label={node[auto, sloped,pos=.15,anchor=south]{#1}}
    },
    for tree={grow=0, child anchor=west, anchor=west, text ragged,
              inner sep=1mm, edge={line width=0.65pt, draw=blue!50}, l sep+=27mm,
              s sep+=5mm, if n children=0{before typesetting nodes={label/.wrap pgfmath arg={right:#1}{content()},
          content={},
          leaf-chance,
        },
      }{},
      edge path={
       \noexpand\path [draw, \forestoption{edge}] (!u.parent anchor) |- (.child anchor)\forestoption{edge label};
       % alternatively, with angled lines
        %\noexpand\path[\forestoption{edge}]
        % (!u.parent anchor) -- ([xshift=-2.6cm].child anchor) --    
        % (.child anchor)\forestoption{edge label};
  },
    }
    [, decision
      [,chance, [{\color{dgreen}\textbf{0}}][{\color{dgreen}\textbf{0}}]]
      [,chance, [{\color{dgreen}\textbf{-1}}][{\color{dgreen}\textbf{5}}]]
      [,chance, [{\color{dgreen}\textbf{11}}][{\color{dgreen}\textbf{-8}}]]
      [,chance, [{\color{dgreen}\textbf{-12}}][{\color{dgreen}\textbf{13}}]]
      ]
    ]
\end{forest}
}
\end{center}
\vspace{3mm}

\begin{tikzpicture}[]
 \put(50,3) {  \node [rectangle, minimum height=8pt, minimum width=8pt, draw=black, fill=black!30!white, thin, inner sep=0pt] at (70,10) {};}
  \put(120,3) {  \node [circle, minimum height=8pt, minimum width=8pt, draw=black, fill=black!30!white, thin, inner sep=0pt] at (70,10) {};}
  \put(190,3) {  \node [isosceles triangle, minimum width=10pt, draw=black, thin, fill=white, inner sep=0pt, shape border rotate=180, outer sep=-\pgflinewidth] at (70,10) {};}   

\end{tikzpicture}
\put(55,5) {Decision Nodes}
\put(125,5) {Event Nodes}
\put(195,5) {Terminal Nodes}
\end{frame}




\end{document}
