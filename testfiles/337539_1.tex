% https://tex.stackexchange.com/a/337539
\documentclass[t,8pt]{beamer}
\usepackage[T1]{fontenc}
\usepackage[ngerman]{babel}
\usepackage[utf8]{inputenc}

\usepackage{tcolorbox}

\setbeamercolor{bgcolorsection}{fg=black,bg=orange!50!red}
\setbeamercolor{bgcolorsubsection}{fg=black,bg=yellow!80!orange}

\setbeamertemplate{frametitle}{%
\ifnum\insertframenumber=\insertsectionstartpage
    \vspace*{0.1mm}
    \begin{tcolorbox}[
        boxrule=0.2mm,
        boxsep=0mm,
        lowerbox=ignored,
        colback=yellow,
        colframe=black
    ]
        \centering
        \Huge\textbf\insertsectionhead\par%
    \end{tcolorbox}
    \vspace*{0.2mm}
\fi%
\ifnum\insertframenumber=\insertsubsectionstartpage
    \vspace*{0.1mm}
    \begin{tcolorbox}[
        boxrule=0.2mm,
        boxsep=0mm,
        lowerbox=ignored,
        colback=yellow,
        colframe=black
    ]
        \centering
        \huge\textbf\insertsubsectionhead\par%
    \end{tcolorbox}
\fi%
\begin{tcolorbox}[
    boxrule=0.2mm,
    boxsep=0mm,
    lowerbox=ignored,
    colback=yellow,
    colframe=black
]
    \centering
    \large\insertframetitle
\end{tcolorbox}
\vspace{-4mm}
}

\begin{document}

\section{Forschungsstand}
\subsection{Aufgabenbeschreibung}

\begin{frame}{Ausgangssituation}
Das ist ein Test
\end{frame}

\subsection{Data Mining}

\begin{frame}{Zum Begriff}

\begin{itemize}
\item \glqq{}Lehre vom \textbf{Sammeln, Säubern, Verarbeiten und Analysieren     von Daten}, um nützliche Erkenntnisse aus ihnen zu gewinnen.\grqq{}
\end{itemize}

\end{frame}


\begin{frame}{}
Test
\end{frame}

\begin{frame}{Datentypen im Bereich Data Mining}
Test
\end{frame}

\begin{frame}{}
\begin{itemize}
\item \textbf{Text Daten:} Auffassen der Daten als multidimensionale Daten,  z.\,B. Analyse von Worthäufigkeiten in einem Text.
\end{itemize}
\end{frame}

\begin{frame}{}

\begin{itemize}
\item \textbf{Räumliche Daten:} Erfassen von Daten an unterschiedlichen Orten (Luftdruck auf Meereshöhe und in den Bergen).
\end{itemize}
\end{frame}

\begin{frame}{Outlier Detection}
Outlier Detection
\end{frame}

\begin{frame}{}
\glqq{}An outlier is an observation which deviates so much from the other   observations as to arouse suspicions that it was generated by a different  mechanism.\grqq{}\footnote{Hawkins, 1980}
\end{frame}

\section{Erstellung der Bachelorarbeit}
\subsection{Ablauf}

\begin{frame}{Zeitplan}
Hier steht der Zeitplan
\end{frame}

\begin{frame}{Gliederung der Bachelorarbeit}
Gliederung
\end{frame}

\begin{frame}{Quellen}
Quellen
\end{frame}

\end{document}
