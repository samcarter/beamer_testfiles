% https://tex.stackexchange.com/a/455292
    \documentclass[xcolor=table]{beamer}


\begin{document}

\begin{frame}
  \frametitle{Conclusion}

\begin{onlyenv}<1-3>
     \begin{block}{Conclusions}
    \begin{itemize}[<+->]
    \item Pour être plus fidèle à la réalité, le modèle peut être amélioré en prenant en

    \vspace{0.15cm}
    considération d’autres aspects physiques (pli d’enveloppe, aéroélasticité. . . ).

    \vspace{0.15cm}
    \item Une analyse par éléments finis sur la structure de l’engin peut être intéressante pour

    \vspace{0.15cm}
    dégager des données aérodynamiques en relation avec les surfaces de contrôle qui

    \vspace{0.15cm}
    peuvent servir pour une commande plus précise et plus avancée.

    \vspace{0.15cm}
    \item Utilisant ces données, une comparaison en simulations du modèle linéarisé et celui non

    \vspace{0.15cm}
    linéaire sera possible et éventuellement fructueuse.
    \end{itemize}
  \end{block}
\end{onlyenv}

\begin{onlyenv}<4->
   \begin{block}{Recommandations}
    \begin{itemize}[<+->]
    \item Pour être plus fidèle à la réalité, le modèle peut être amélioré en prenant en

    \vspace{0.15cm}
    considération d’autres aspects physiques (pli d’enveloppe, aéroélasticité. . . ).

    \vspace{0.15cm}
    \item Une analyse par éléments finis sur la structure de l’engin peut être intéressante pour

    \vspace{0.15cm}
    dégager des données aérodynamiques en relation avec les surfaces de contrôle qui

    \vspace{0.15cm}
    peuvent servir pour une commande plus précise et plus avancée.

    \vspace{0.15cm}
    \item Utilisant ces données, une comparaison en simulations du modèle linéarisé et celui non

    \vspace{0.15cm}
    linéaire sera possible et éventuellement fructueuse.
    \end{itemize}
  \end{block}
\end{onlyenv}
\end{frame}
\end{document} 
