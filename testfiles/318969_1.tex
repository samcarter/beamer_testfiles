% https://tex.stackexchange.com/a/318969
\documentclass[12pt, aspectratio=169, noamssymb]{beamer} 
\usetheme{default}
\setbeamertemplate{navigation symbols}{}
\usepackage[T1]{fontenc}
\setbeamercolor{background canvas}{bg=black}
\setbeamercolor{normal text}{fg=white}
\usepackage[utf8]{inputenc}
\usepackage[norsk]{babel}
\usepackage{amsmath}
\newtheoremstyle{newstyle}
{3000pt}   % ABOVESPACE
{3000pt}   % BELOWSPACE
{\normalfont}  % BODYFONT
{0pt}       % INDENT (empty value is the same as 0pt)
{\bfseries} % HEADFONT
{.}         % HEADPUNCT
{5pt plus 1pt minus 1pt} % HEADSPACE
{}          % CUSTOM-HEAD-SPEC
\theoremstyle{newstyle}
\newtheorem{teorem}{\normalsize \color{white} \bfseries Teorem}
\newtheorem{korollar}{\normalsize \color{white} \bfseries Korollar}

\usepackage{etoolbox}
\addtobeamertemplate{theorem begin}{%
    \vspace{0.5cm}
}

\addtobeamertemplate{theorem end}{%
    \vspace{0,5cm}
}


\begin{document}

    \begin{frame}
        \begin{teorem}[Skjæringssetningen]
            La $X$ være et koblet og $Y$ er en ordnet mengde i ordentopologien. La $f: X \to Y$ være kontinuerlig. Hvis $a,b \in X$ og $r \in Y$ slik at $f(a) <r < f(b)$, da eksisterer $c \in X$ slik at $f(c) = r$. 
        \end{teorem}
        \begin{korollar}
            Anta at $f:[a,b] \to R$ er kontinuerlig og $r \in R$ slik at $f(a) <r < f(b)$, da eksisterer $c \in [a,b]$ slik at $f(c) = r$. 
        \end{korollar}
    \end{frame}

\end{document}
