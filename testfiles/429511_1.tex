% https://tex.stackexchange.com/a/429511
\documentclass[9pt,t,handout]{beamer}

\usepackage[utf8]{inputenc}
\setbeamertemplate{navigation symbols}{}

\setbeamerfont{author in head/foot}{size=\footnotesize}
\setbeamerfont{author}{size=\small}
\setbeamerfont{institute}{size=\footnotesize}
\setbeamerfont{date}{size=\footnotesize}
\setbeamerfont{normal text}{size=\footnotesize}
\AtBeginDocument{\usebeamerfont{normal text}}

\setbeamertemplate{itemize item}[circle]

\definecolor{myblue}{rgb}{0.196,0.196,0.694}
\definecolor{myred}{rgb}{1,0.3,0.1} 

\setbeamercolor{alerted text}{fg=myred}
\setbeamercolor{structure}{fg=myblue}

\author{Max Mustermann}
\institute[Univ. of Berlin]{Seminar zur Numerik im SS 2018, Universit\""at of Berlin}
\title{Sobolev-Orthogonalpolynome}
\date{15.Mai 2018}

\begin{document}

\begin{frame}
    \maketitle

  \pause

  \alert{Ziel} dieses Vortrags: Motivation der Thematik 
  \begin{itemize}
    \item Sobolev-Orthogonalpolynome: Theorie und Anwendungen aus   [G1]
    \item Sobolev-Orthogonalpolynome: Verwendung der Matlab-Programme aus [G2]
  \end{itemize}

    \vfill
    \begingroup
        \tiny
    \alert{Verwendete Literatur} (zus\""atzlich zu Originalarbeiten): 
    \begin{itemize}
        \item[{[G1]}] W. Gautschi, Orthogonal Polynomials, Oxford University    Press, 2004.
        \item[{[G2]}] W. Gautschi, Orthogonal Polynomials in Matlab, SIAM, 2016.
    \end{itemize}
  \endgroup
\end{frame}

\begin{frame}
    \structure{Definition} 

  \begin{itemize}
      \item 
      \item $\ldots$
  \end{itemize}

  \structure{Definition} 
  \begin{itemize}
      \item
      \item
      \item
      \item
      \item
      \item $\ldots$
  \end{itemize}

\end{frame}

\begin{frame}

    \begin{center}
         \alert{Motivation}

         \alert{Definition \& Eigenschaften}
    \end{center}

  \begin{itemize}
    \item Sei $\mathbb{P} :=\{relle \quad Polynome\}$ für ein Paar $u,v \in    \mathbb{P}$ definieren wir 
      $(1.1) (u,v)_S= \int_{\mathbb{R}}u(t)v(t) d\lambda_0(t)+\cdots+\int_{\mathbb{R}}u^{(s)}(t)v^{(s)}(t) d\lambda_s(t)$ wobei $s\le 1$ $d\lambda_i$ positive  Maße (mit nicht notwendigerweise selben Trägern) als Sobolev inneres  Produkt
  \end{itemize}

  \pause

    \begin{center}
         \alert{Motivation}

         \alert{Definition \& Eigenschaften}
    \end{center}

  \begin{itemize}
      \item Die Norm ist definiert durch $\lVert u\rVert_S=\sqrt{(u,u)_S}=    \sqrt{\sum_{\sigma=1}^s\int_{\mathbb{R}}(u^{(\sigma)}(t))^2} d\lambda_{\sigma}(t)$\\ sofern $s\le 1$, erfüllt $(1.1)$ nicht länger die Shift-Eigenschaft.
    Für $s=1$, z.B. ist $(tu,v)_S=(u,tv)_S=\int_{\mathbb{R}}(uv'-u'v)(t) d\lambda_1(t)$ 
    Wähle $u(t)=1$ und $v(t)=t$ $\Rightarrow \int_{\mathbb{R}}d\lambda_1(t)>0$
  \end{itemize}

\end{frame}

\end{document}
