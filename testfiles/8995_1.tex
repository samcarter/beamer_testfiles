% https://texnique.fr/osqa/questions/8995/beamer-insertion-dun-graphique-a-gauche-et-de-quatre-images-a-droite/8997
\documentclass[french,9pt,t]{beamer}
\usetheme{JuanLesPins}
\setbeamercovered{transparent}
\definecolor{bleuturquoise}{cmyk}{1.0,0.3,0.0,0.15}
\definecolor{Firebrick3}{rgb}{0.804,0.150,0.150}
\usecolortheme[named=bleuturquoise]{structure}

\usepackage[T1]{fontenc}
\usepackage{lmodern}
\usepackage[french]{babel}
\setbeamersize{text margin left = 2mm, % normalement c'est 1 cm
    text margin right = 2mm, % normalement c'est 1 cm
}

%\usepackage{color}
%\usepackage{xcolor}
% \usepackage[dvipsnames,svgnames]{xcolor}
\usepackage{pgf}
\usepackage{tikz}
%\usepackage{graphicx}
%\DeclareGraphicsExtensions{.jpg,.pdf,.png}
\usepackage{caption}
%\usepackage{amssymb}
%\usepackage{amsmath}
%\usepackage{amsfonts}
%\usepackage{url}
%\usepackage{setspace}
\usepackage{gensymb}
\usepackage{mathrsfs}
\usepackage{chemfig}
\usepackage{pgfplots}
\usepackage{siunitx}
\usepackage{booktabs}
\usepackage{tabularx}
\usepackage{array}
% \usepackage{ragged2e, makecell, mhchem}
%\usepackage{fancyhdr}
%\pagestyle{fancy}
%\pagestyle{empty}
\usepackage{chemformula} % pour les formules chimiques
\pgfplotsset{compat=newest} % pour bénéficier des dernières fonctionnalités de pgfplots
%\usepackage{caption}

\makeatletter
\newcommand\fcaption{\captionsetup{font=small}\def\@captype{figure}\caption}
\newcommand\tcaption{\captionsetup{font=small}\def\@captype{table}\caption}
\makeatother

\setbeamertemplate{headline}{}

\begin{document}

\begin{frame}%[shrink=20]
  \frametitle{R. $\kappa$ en fonction de la durée de SVA pour 0206 et 0206-Au}
  \begin{columns}[onlytextwidth,c]
    \begin{column}{.55\textwidth}
      \begin{tikzpicture}
        \begin{axis}[
          xlabel={durée de SVA (min)},
          ylabel={longueur de corrélation $\kappa$ (nm)},
          axis lines=left,grid=major,xmin=0,xmax=1050,ymin=0,ymax=1170,
          log ticks with fixed point,
          width=\textwidth,
          legend style={at={(0.2,1.00)},anchor=north,legend columns=1}
        ]
          \addplot +[bleuturquoise,mark size=2pt,mark=*,thick,smooth,mark options={fill=bleuturquoise},error bars/.cd,y dir=both,y explicit,error bar style={thick}] coordinates {(0,68)+-(0,30)(100,125)+-(0,30)(300,164)+-(0,30)(400,148)+-(0,30)(1000,545)+-(0,30)};
          \addplot +[Firebrick3,mark size=3pt,mark=*,mark options={fill=Firebrick3},thick,smooth,error bars/.cd,y dir=both,y explicit,error bar style={thick}] coordinates {(120,148)+-(0,43)(300,203)+-(0,31)(400,194)+-(0,38)(1000,1035)+-(0,107)};
          \legend{20200206,20200206-Au};
        \end{axis}
      \end{tikzpicture}
    \end{column}
    \begin{column}{.4\textwidth}
      \includegraphics[clip,scale=0.14]{example-image-duck}
      \fcaption{0206-Au - SVA120}
      \includegraphics[clip,scale=0.25]{example-image-duck}
      \fcaption{0206-Au - SVA300}
      \includegraphics[clip,scale=0.05]{example-image-duck}
      \fcaption{0206-Au - SVA400}
      \includegraphics[clip,scale=0.05]{example-image-duck}
      \fcaption{0206-Au - SVA1000}
    \end{column}
  \end{columns}
\end{frame}

\end{document}