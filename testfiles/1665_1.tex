% https://topanswers.xyz/tex?q=1423#a1665
\documentclass[xcolor={rgb}]{beamer}
\beamertemplatenavigationsymbolsempty
\setbeamersize{text margin left=10mm,text margin right=5mm} 
\setbeamertemplate{frametitle}[default][center]
\usepackage[T1]{fontenc}
\usepackage{lmodern}
\usepackage{bm}
\usepackage{mathtools}
\usepackage[many]{tcolorbox}
\usepackage{adjustbox}
\usepackage{listings}
\lstdefinestyle{duckstyle}{%
    moredelim=[is][\color{red}]{|}{|},
    mathescape=true,
    escapechar=@,
    basicstyle=\ttfamily,
    columns=fullflexible
}
\lstset{style=duckstyle}
\newcommand{\ubar}[1]{\alt<+>{\underaccent{\bar}{#1}}{#1}}
\tcbuselibrary{skins}
\tcbset{
	arc=0pt,
	outer arc=0pt,
	colback=white,
}

\usepackage{circledsteps}
\usepackage{tikz}
\usetikzlibrary{matrix}
\newcommand{\mycircle}[1][1]{%
\Circled[fill color={foo!![#1]},outer color=blue!60]{\phantom{\mbox{o}}}}%
\def\numcolors{4}
\pgfmathparse{1/\numcolors}%
\definecolorseries{foo}{hsb}{step}{red!90!black}{\pgfmathresult,0,0} 
\resetcolorseries[\numcolors]{foo}%
\makeatletter
\newcommand{\UncoverList}[2][]{\c@pgf@counta0\relax
\pgfutil@for\my@item:={#2}\do{%
\advance\c@pgf@counta by1%
\only<\c@pgf@counta->{\my@item}%
}}
\makeatother

\usetikzlibrary{arrows.meta,
	shapes,
	tikzmark}
\usetikzlibrary{tikzmark,shapes.geometric, patterns,decorations.pathreplacing, calligraphy}

\usetikzmarklibrary{listings} 


\newcommand{\var}[1]{\operatorname{var}{\!#1}} 


\newcommand{\tikzcircle}[3][red]{\tikz[baseline=-0.5ex]{\node[draw={#1}, fill={#1}, circle,text width=#2,minimum width=1.4em,text=black,align=center] at (0,0) {#3};}\hspace{0pt}}%

\pgfmathsetseed{2}

\setbeamertemplate{itemize items}{}

\begin{document}


\begin{frame}[fragile]
\frametitle{frame}
{\centering
\begin{tikzpicture}

\node (crow) {\vphantom{\raisebox{1ex}{\mycircle[1]}}\def\numdots{12}%
\def\cols{1,2,1,4,1,2,3,1,1,2,3,4}
\foreach \x [count=\y from 2] in {1,...,\numdots}{%
  \pgfmathtruncatemacro{\mycf}{{\cols}[\x-1]}%
  \ifnum\y>1\relax\;\fi\alt<\y>{%
  	\raisebox{1ex}{\mycircle[\mycf]}}{%
	\mycircle[\mycf]}%
}%
};

\end{tikzpicture}
}

\begin{visibleenv}<2->
\begin{tcolorbox}[top=0pt, left=5pt,right=5pt, colback=blue!5!white, text height=4.8cm]
\begin{lstlisting}[ mathescape, name=foo, basewidth = {.3em}]
stream $\langle a_1, \dots, a_m \rangle$,$a_i \in [n]$
\end{lstlisting}
\end{tcolorbox}
\end{visibleenv}%

\end{frame}

\end{document}  