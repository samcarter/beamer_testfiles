% https://tex.stackexchange.com/a/403595
\documentclass{beamer} 

\usepackage[brazilian,hyperpageref]{backref}     % Paginas com as citações na bibl

\usepackage{stmaryrd}
%\usepackage{graphicx}

\usetheme{Antibes}
\usecolortheme{beaver}

%    \justifying

\usepackage{parskip}
\setlength{\parskip}{\smallskipamount}

\usepackage[utf8]{inputenc}
\usepackage[brazil]{babel}
\usepackage[T1]{fontenc}
\usefonttheme{serif}
\usepackage{lipsum}
%\usepackage{caption}
\setbeamercolor{block title}{bg=red!30,fg=black}
\usepackage{times}
\usepackage{tikz}
\usepackage{amsmath}
\usepackage{verbatim}
\usepackage{multirow}
\usepackage{framed} 
\usepackage[bottom]{footmisc}
\usepackage{arydshln}

%\usetikzlibrary{arrows,shapes}
\usetikzlibrary{arrows,shapes,positioning,shadows,trees}
%\usetikzlibrary{trees}
%\usepackage{tikz}
%\usetikzlibrary{positioning}
\usetikzlibrary{arrows.meta,chains}
%\usetikzlibrary{arrows,shapes,positioning,shadows,trees}

%\usepackage{float}
%\usepackage{caption}


\author{\texorpdfstring{Gabriel Petrini da Silveira \and RA 155468
\\ \textbf{Orientador:} Antônio Carlos Macedo e Silva \\ 
{\footnotesize\ttfamily gpetrinidasilveira@gmail.com}}{Gabriel Petrini da Silveira}}
\title[Política Fiscal: HPE, Grande Recessão e SFC]{Política Fiscal e(m) Grande Recessão: uma análise com Consistência entre Fluxos e Estoques}
\institute{Instituto de Economia - UNICAMP}
\date{05 de Dezembro - 2017}

%\makeindex

\tikzset{
    basic/.style  = {draw, text width=2cm, font=\sffamily, rectangle},
    root/.style   = {basic, rounded corners=2pt, thin, align=center,
        fill=green!30},
    level 2/.style = {basic, rounded corners=6pt, thin,align=center, fill=green!60,
        text width=8em},
    level 3/.style = {basic, thin, align=left, fill=white, text width=6.5em}
}



\begin{document}

\begin{frame}  
\frametitle{Características do NCM}     

Neste arcabouço:
\begin{description}         
    \item[Principal instrumento] Taxa de juros de curto prazo       
    \item[Aparato institucional] Regime de Metas para a Inflação        
    \item[Objetivo] Estabilização da inflação
\end{description}

\begin{figure}%[H]
    \begin{tikzpicture}[
    > = {Latex[]},
    start chain = going right,
    node distance=7mm,
    block/.style={shape=rectangle, draw,
        inner sep=1mm, align=center,
        minimum height=7mm, on chain}]     %Para que servem esses comandos?
    %placing the blocks
    \node[block] (n1) {Política monetária \\ eficiente};
    \node[block] (n2) {$Y^G$ estável};
    \node[block] (n3) {Inflação estável};
    \node[block] {Nível de \\ atividade};

    \draw[->] (n1.east) --  + (0,0mm) -> (n2.west);
    \draw[->] (n2.east) --  + (0,0mm) -> (n3.west);
    \draw[<->] (n1.south) --  + (0,-7mm) -| (n3.south);
    \draw[<->] (n2.north) --  + (0,+7mm) -| (n3.north);
    \end{tikzpicture}
 \end{figure}

 \begin{framed}
    Portanto, há uma hierarquia entre os objetivos de política econômica, subordinando os demais ao controle da inflação.
 \end{framed}

\end{frame}

\begin{frame}
\frametitle{SFC e política fiscal}
\framesubtitle{Estrutura Modelo G\&L}
 \begin{center}
    \begin{tikzpicture}[
    > = {Latex[]},
    start chain = going right,
    node distance=7mm,
    block/.style={shape=rectangle, draw,
        inner sep=1mm, align=center,
        minimum height=7mm, 
        join=by ->, on chain}]     %Para que servem esses comandos?
    %placing the blocks
    \node[block] (n10) {Modelo SFC-PK};
    \node[block] (n11) {Endogenização \\ dos gastos \\ públicos};
    \node[block] (n12) {Função de reação};
 %  \node[block] (n4) {Endogenização \\ dos gastos \\ públicos};
 %  \node[block] (n5) {Função de reação};
    \end{tikzpicture}
    %Fim Fluxograma

    \begin{equation}
    \label{FiscalReac}
    gr_G = gr - \beta_1\cdot \Delta \pi_{-1} - \beta_2\cdot (\pi_{-1} - \pi^T)
    \end{equation}

    \begin{tikzpicture}[
 > = {Latex[]},
 start chain = going right,
 node distance=7mm,
 block/.style={shape=rectangle, draw,
    inner sep=1mm, align=center,
    minimum height=7mm,  on chain}]     %Para que servem esses comandos?
 %placing the blocks
 \node[block] (n13) {NCM};
 \node[block] (n14) {Função de reação};
 \node[block] (n15) {$\Delta \pi, y$};
 %  \node[block] (n4) {Endogenização \\ dos gastos \\ públicos};
 %  \node[block] (n5) {Função de reação};
 \draw[->] (n13.south) --  + (0,-7mm) -| (n14.south);
 \draw[->] (n14.north) --  + (0,+7mm) -| (n13.north);
 \draw[->] (n14.east) --  + (0,0mm) -> (n15.west);
 \end{tikzpicture}
 \end{center}
 \end{frame} 
\end{document}
