% https://topanswers.xyz/tex?q=1290#a1525
\documentclass{beamer}
\beamertemplatenavigationsymbolsempty
\setbeamersize{text margin left=10mm,text margin right=5mm} 
\setbeamertemplate{frametitle}[default][center]
\usepackage{tikz}

\newcommand{\tikzcircle}[2][red]{\tikz[baseline=-0.5ex]{\fill[{#1},radius=#2] (0,0) circle ;}\hspace{0pt}}%

\def\numcolors{6}
\def\numdots{100}

\pgfmathparse{1/\numcolors}%
\definecolorseries{foo}{hsb}{step}{red!90!black}{\pgfmathresult,0,0} 
\resetcolorseries[\numcolors]{foo}% 

\newcounter{bar}
\foreach \x in {1,...,\numcolors}{
  \setcounter{bar}{\x}
  \newcounter{mycount\alph{bar}}
}

\pgfmathsetseed{\number\pdfrandomseed}

\begin{document}
\begin{frame}{The counting problem}
\foreach \x in {1,...,\numdots}{%
  \pgfmathrandominteger{\myran}{1}{\numcolors}%
  \setcounter{bar}{\myran}%
  \addtocounter{mycount\alph{bar}}{1}%
  \tikzcircle[{foo!![\myran]}]{8pt}%
}%
\end{frame}

\begin{frame}{The counting problem}
\foreach \x in {1,...,\numcolors}{%
  \setcounter{bar}{\x}%
  \foreach \y in {1,...,\csname themycount\alph{bar}\endcsname}{%
    \tikzcircle[{foo!![\x]}]{8pt}%
  }%
}
\end{frame}

\end{document}