% https://topanswers.xyz/tex?q=1775#a2000
\documentclass{beamer}
\usepackage[beamer,customcolors]{hf-tikz}
\usepackage{tikz}
\usepackage{tcolorbox}
\usetikzlibrary{tikzmark, chains, fit,
	shapes}
  \usetikzlibrary{backgrounds,calc}
\begin{document}

\begin{frame}
\frametitle{}
\begin{tabular}{lll} 
& IIIIII-IIII & IIIIII-DDDDDD  \\
\hline CCCCCCCC & DDDDDDDDDDDDD & RRRRRRRRRR \\
\tikzmarkin<2->{a}Bipartiteness\tikzmarkend{a} & BBBBBBBBBBBBB & RRRRRRRRRR \\
CCCCCCCCCC & DDDDDDDDDDDDD & RRRRRRRRRR  \\
SSSSSSSSSSSSS & DDDDDDDDDDDDD & RRRRRRRRRR \\
& MMMMMMM & OOOOOOOO \\
MMMMMMMM & MMMMMMMMMM & OOOOOOOOOO \\
MMMMMMMM & MMMMMMMMMM  \\
& & EEEEEEEEEE \\
 UUUUUUUU  & MMMMMMM& OOOOOOOO\\
WWWWWWWW & MMMMMMMMMM & OOOOOOOO\\
\hline
\end{tabular}

\begin{visibleenv}<2->
\definecolor{myblue}{RGB}{80,80,160}
\definecolor{mygreen}{RGB}{80,160,80}
\vspace*{-4cm}\hspace*{4cm}
\begin{tikzpicture}[thick,
  every node/.style={draw,circle},
  fsnode/.style={fill=myblue},
  ssnode/.style={fill=mygreen},
  every fit/.style={ellipse,draw,inner sep=-2pt,text width=2cm},
  ->,shorten >= 3pt,shorten <= 3pt,
  scale=0.3,
  background rectangle/.style={fill=white, draw=black, rounded corners, inner frame sep=1cm }, 
  show background rectangle,
  remember picture,
]

\begin{scope}[start chain=going below,node distance=3mm]
\foreach \i in {1,2,...,5}
  \node[fsnode,on chain] (f\i)  {};
\end{scope}

\begin{scope}[xshift=4cm,yshift=-0.5cm,start chain=going below,node distance=3mm]
\foreach \i in {6,7,...,9}
  \node[ssnode,on chain] (s\i)  {};
\end{scope}

\node [myblue,fit=(f1) (f5)] {};

\node [mygreen,fit=(s6) (s9)] {};

% the edges
\draw (f1) -- (s6);
\draw (s6) -- (f2);
\draw (f2) -- (s7);
\draw (s7) -- (f3);
\draw (s8) -- (f3);
\draw (f3) -- (s9);
\draw (s9) -- (f5);
\draw (f5) -- (s6);

\draw<2>[red,overlay] (current bounding box.west) -- (a);

\end{tikzpicture}
\end{visibleenv}
\end{frame}
\end{document}