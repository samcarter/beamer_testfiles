% https://tex.stackexchange.com/a/326602
\documentclass[french,,aspectratio=1610]{beamer} 
\usepackage{tcolorbox} 


% Boite pour les theorèmes : 
\usepackage{tcolorbox} 
\tcbuselibrary{skins,breakable} 

% Théorème, proposition, lemme 

\newenvironment{theo}[1][]{% 
\tcolorbox[noparskip,% 
            breakable,% 
            colframe=blue,colback=blue!20!white,% 
            coltitle=black,% 
            fonttitle={\bfseries \scshape} ,% 
            title=Théorème :  #1]}% 
            {\endtcolorbox}  
%%%%%% Habituel pour compiler 
% =========================== 
\usepackage[latin1]{inputenc} 
\usepackage[T1]{fontenc} 
\usepackage[upright]{fourier} 


\begin{document} 
    \begin{frame} 
            \begin{theo}[Mon titre] 
                    Mon premier \only<.(2)->{Mon deuxième} 
            \end{theo} 

                        \begin{onlyenv}<3->
                \begin{theo} 
                                \only<3->{Mon troisième} \only<4->{mon quatrième} 
                \end{theo}  
                        \end{onlyenv}                   


    \end{frame} 
\end{document} 
